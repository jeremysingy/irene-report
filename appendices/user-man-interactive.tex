
\chapter{Interactive Tracking User Manual}
\label{chap:usermanint}

\section{Introduction}

This documents aims to explain the new \emph{Interactive Tracking} feature developed during this thesis, from the end-user's point of view. It will details the steps required to use all new functions and to be able to recover a record using the new features.

As a prerequisite, the user should already be familiar with the existing PRISM and/or RENE programs to restore recordings acquired with IRENE. This document is focused on the version as integrated in PRISM. Almost explanations on this document also apply to the RENE version. The differences are explained in \autoref{sec:inttrackrene}.

\section{General description}

\subsection{User interface overview}

In the application user interface, the main difference is the addition of a new panel on the right side, where the detailed record view also stands (see \autoref{fig:intpanel}. It is accessible by clicking on the \emph{Interactive Tracking} tab. The detailed view as already existing remains unchanged, and one can switch between the two views at any time. A new tracking option has also been added at the list.

\begin{figure}[!ht]
\centering
\includegraphics[width=0.9\textwidth]{images/int-right-panel}
\caption[Location of the interactive panel in PRISM.]
{Location of the interactive panel in PRISM as well as the option highlighted in orange.}
\label{fig:intpanel}
\end{figure}

Almost all the new features are located on this panel. The upper part contains the controls and some information about the record being tracked. The remaining space is used to draw the record while tracking. When no record is loaded, this part is blank except for a red arrow.

A description of this panel can be seen in \autoref{fig:intguicontrols}. Specific controls are also detailed in \autoref{tab:intguicontrols}.

\begin{figure}[!ht]
\centering
\includegraphics[width=0.6\textwidth]{images/int-track-controls}
\caption{Visualization of the controls in the panel.}
\label{fig:intguicontrols}
\end{figure}

\begin{table}[h!]
\begin{center}
\tabulinesep=3pt
\begin{tabu} to 0.7\textwidth {| c | X[m] |} %{>{\bfseries}lX}
    \everyrow{\hline}
    \hline
    \rowfont[c] \bfseries
        No. & Description \\
        1 & Text box to change the height for bitmap contrast normalization (see \autoref{sec:contrastnorm}) \\
        2 & Tracking speed selector \\
        3 & Use curve fitting to make tracked position more smooth accurate. \\
        4 & Zoom control \\
        5 & Indication of the current position pointed by the red arrow \\
        6 & Start/Stop tracking button \\
        7 & Tracking controls (see below) \\
        8 & Arrows indicating the currently tracked position \\
\end{tabu}
\end{center}
\caption{Transformation of the different objects in the interactive panel.}
\label{tab:intguicontrols}
\end{table}

\subsection{Description of controls}

The upper part is enables to control different parameters for the tracking. The first one, Norm. height may may remain unchanged most of time. Its behavior is detailed in \autoref{sec:contrastnorm}. The \emph{Y speed} is for setting the speed while tracking a record. One can directly set the value or increment it. While tracking the record, it is also possible to increment and decrement the speed with the mouse wheel, without stopping the movement.

The option \emph{Use curve fit} enables to improve the tracking by adapting the center of the groove using parabola fitting. If unchecked, the smallest value in the neighborhood is defined as the real center.

To change the zoom level, whether to see more details or an overview of the scan, one can slide the \emph{Zoom} control. If the check box \emph{Locked with speed} is enabled, the zoom is automatically adjusted according to the tracking speed. This can be useful when a specific part needs more accurate tracking, slowing down will directly show more details.

By default, the tracking is not active. To actually follow the groove, the \emph{Start/Stop} button must firstly be pressed. The tracking can be stopped at any time, e.g. to look over other parts and continue later.

\section{Loading a new acquisition}

Before loading the acquisition, the first step is to select the proper option, \emph{Interactive Tracking}, in the list of tracking types. Then, it remains to select the needed usual options for loading and processing and load the acquisition file into PRISM.

Once the record has been loaded, if the interactive tracking was selected, the image heightmap is directly viewed on the interactive panel, with the arrow pointing at the top-left corner.

\subsection{Contrast normalization}
\label{sec:contrastnorm}

The only option that must be set before loading the disc is \emph{Norm. height}. In some cases, the absolute surface height varies a lot on big areas, making the structure of grooves negligible and almost invisible if the contrast was normalized from the entire range of values (for more information, see \autoref{sec:contrastadj}). The original detailed view in PRISM is not directly concerned by this issue, as the contrast is readjusted each time the user selects a new area.

However, recomputing it would be too slow to draw the record interactively. Therefore, the normalization is performed in smaller area. The width corresponds to the width of a pass and the height may be controlled with this option. However, a value of 60 should be adequate in most of situations.

\section{Tracking interactively}

This section details all options available to the user to help manually tracking a record as quickly and correctly as possible.

\subsection{Piloting the tracking}

%\footnote{The record automatically stays in the limit}.
In the interactive panel, the drawn record can be seen as a draggable object. When the mouse is pressed, it is possible to shift it horizontally by moving the cursor. The red arrow represents the currently tracked position and remains fixed at the top-center. It should therefore point at the center of the groove. An example of starting position can be seen in \autoref{fig:intshift}.

\begin{figure}[!ht]
\centering
\includegraphics[width=0.7\textwidth]{images/int-mouse-shift}
\caption[Example of starting position.]
{Example of starting position. The record is dragged to the wanted position. The vertex of the arrow represents the position to be tracked. Here the tracking is not active, as the \emph{Start/Stop} button has not been pressed. In this specific case, the starting groove is located on the right (tracking from right to left).}
\label{fig:intshift}
\end{figure}

Once the groove as been selected, the tracking may start. To actually save the current tracking points, the \emph{Start/Stop} button must be pressed. Then to start tracking, the most convenient way is to click on the record image and add speed using the mouse wheel. The image then starts scrolling down, representing the record being tracked.

The main goal is then to keep the top of the arrow as close as possible to the groove center, ensuring a proper tracking. The speed can be adjusted at any time with the mouse wheel. The user can also release the click, change the speed on the control and click again on the record to continue tracking.

\subsubsection{Real tracked position}

%\subsubsection{Visualization on binned image}
\subsection{On the binned image}

At any time, the current position can be viewed globally on the binned image as a little red square. This is very useful to locate the current position on the whole record. Another important feature is that it is also possible to click on the binned image to move to the clicked position. If the tracking is enabled at this moment, this will also add a new tracking point. In fact, it will act the very same way as the former manual tracking. This is very useful when a specific area does not require a very precise analysis, enabling to track it very quickly.

The two views are then always synchronized, each of them giving information at another level. The current path representing the part already tracked is also represented in orange. \autoref{fig:inttracksync2} represents an example of the part of a groove tracked and visualized on both views.

\begin{figure}[!ht]
\centering
\includegraphics[width=1.0\textwidth]{images/int-track-sync-2}
\caption[A groove being tracked.]
{A groove being tracked. The two positions are synchronized and the path is drawn in orange.}
\label{fig:inttracksync2}
\end{figure}

\subsubsection{A magnifier tool}

When the cursor is moving over the binned image (without clicking on it) the portion viewed in the interactive panel is temporarily moved accordingly. This then acts as an interactive magnifier (adjustable with the zoom), which may be useful on a former analysis step or while tracking, e.g. to find the best matching when a crack appears.

\subsection{Start a new revolution}

When an entire revolution has been performed, the final position is at the bottom of the mapped image. If the tracking is correct, the tracking may continue directly at the same horizontal position. The system is able to move back by setting a negative speed. However, to simplify the case of setting the position to the top, a button has been added for this purpose. In reality, it does three things:

\begin{itemize}
\item Go back to the top (without changing horizontal position)
\item If tracking active, add a point at this position (representing the starting point of the new revolution)
\item Sets the speed to zero.
\end{itemize}

Resetting the speed is useful so that the position can be adjusted before continuing. The tracking can then start easily from this point. One can note that the tracking is drawn not only in the binned image, but also on the interactive panel, which gives a more detailed view of the result. It is also useful to compare the current track to the previous revolution. On the binned view, the link from the end to the start of the next revolution (conceptually the same position) is draw in red, as viewed in \autoref{fig:intnewrev}.

\begin{figure}[!ht]
\centering
\includegraphics[width=1.0\textwidth]{images/int-new-rev}
\caption{Example when another revolution as started.}
\label{fig:intnewrev}
\end{figure}

\subsection{Let the program do it}

Manual or automatic tracking are mainly useful with heavily damaged records that cannot be tracked with an automatic algorithm. However, while it is not directly possible to track over a crack for example, it could be possible to let the program find the way until an irregularity occurs.

The interactive tracking is able to automatically track a section from the current position to a specified one. The user must simply press the \emph{Shift} key while clicking on the record. This way, it is not a straight line that is drawn but the real center is followed until the current position.

\subsection{Correct an error}

TODO.

\subsection{Copy an existing pattern}

TODO.

\section{Starting record processing}

TODO.

\section{Interactive tracking in RENE}
\label{sec:inttrackrene}

TODO.