% Abstract

\vspace*{140pt}
\begin{abstract}
\thispagestyle{plain}

Recovery of old records using optical technologies is used for years by Carl Haber and his team at Lawrence Berkeley National Laboratory (LBNL). So far, a lot of them have already been restored using systems based on 2D or 3D scanning for the acquisition and programs that use the result to get the original sound. However, some of these records are heavily damaged making them difficult to process. A common issue is when the lacquer layer shrinks resulting in so-called cracked records. This project aims to find solutions in order to process different types of these damaged records, whether by improving user input and control or by proposing a more automatic way. The results have shown that a fully-automated processing is hardly practical on a heterogeneous set of records but works in specific situations, though still relying on human decision for matchings. For the most difficult cases, the user may rely on improved manual features.

%\smallskip
\vspace{1cm}
\noindent
\textbf{Keywords:} Image and sound processing, Gramophone records, LBNL, Programming
\end{abstract}